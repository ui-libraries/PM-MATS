\chapter*{❋2. Immediate Consequences of the Primitive Propositions.} %\addcontentsline{toc}{section}{❋2. Immediate Consequences of the Primitive Propositions.} 
\pageFsp{102}{98} \textit{Summary of $\pmast$2.}

The proofs of the earlier of the propositions of this number consist simply in noticing that they are instances of the general rules given in $\pmast$1. In such cases, these rules are not premisses, since they assert any instance of themselves, not something other than their instances. Hence when a general rule is adduced in early proofs, it will be adduced in brackets\footnote{Later on we shall cease to mark the distinction between a premiss and a rule according to which an inference is conducted. It is only in early proofs that this distinction is important.}, with indications, when required, as to the changes of letters from those given in the rule to those in the case considered. Thus ``$\pmSUb{\text{Taut}}{\pmnot p}{p}$'' will mean what ``Taut'' becomes when $\pmnot p$ is written in place of $p$. If ``$\pmSUb{\text{Taut}}{\pmnot p}{p}$'' is enclosed in square brackets before an asserted proposition, that means that, in accordance with ``Taut," we are asserting what ``Taut" becomes when $\pmnot p$ is written in place of $p$. The recognition that a certain proposition is an instance of some general proposition previously proved or assumed is essential to the process of deduction from general rules, but cannot itself be erected into a general rule, since the application required is particular, and no general rule can \textit{explicitly} include a particular application.

Again, when two different sets of symbols express the same proposition in virtue of a definition, say $\pmast1\pmcdot01$, and one of these, which we will call (1), has been asserted, the assertion of the other is made by writing ``$[(1)\pmdot(\pmast1\pmcdot01)]$" before it, meaning that, in virtue of $\pmast1\pmcdot01$, the new set of symbols asserts the same proposition as was asserted in $(1)$. A reference to a definition is distinguished from a reference to a previous proposition by being enclosed in round brackets.

The propositions in this number are all, or nearly all, actually needed in deducing mathematics from our primitive propositions. Although certain abbreviating processes will be gradually introduced, proofs will be given very fully, because the importance of the present subject lies, not in the propositions themselves, but (1) in the fact that they follow from the primitive propositions, (2) in the fact that the subject is the easiest, simplest, and most elementary example of the symbolic method of dealing with the principles of mathematics generally. Later portions---the theories of classes, relations, cardinal numbers, series, ordinal numbers, geometry, etc.---all employ the same method, but with an increasing complexity in the entities and functions considered.

\pageFsp{103}{99} The most important propositions proved in the present number are the following:
\begin{flalign*}
	& \boldsymbol{\pmast2\pmcdot02}. \quad \pmthm \pmdott q \pmdot \pmimp \pmdot p \pmimp q &
\end{flalign*}

\textit{I.e.}\ $q$ implies that $p$ implies $q$, \textit{i.e.}\ a true proposition is implied by any proposition. This proposition is called the ``principle of simplification'' (referred to as ``Simp''), because, as will appear later, it enables us to pass from the joint assertion of $q$ and $p$ to the assertion of q simply. When the special meaning which we have given to implication is remembered, it will be seen that this proposition is obvious.
\begin{flalign*}
	& \boldsymbol{\pmast2\pmcdot03}. \quad \pmthm \pmdott p \pmimp \pmnot q \pmdot \pmimp \pmdot q \pmimp \pmnot p & \\
	& \boldsymbol{\pmast2\pmcdot15}. \quad \pmthm \pmdott \pmnot p\pmimp q \pmdot \pmimp \pmdot \pmnot q \pmimp p & \\
	& \boldsymbol{\pmast2\pmcdot16}. \quad \pmthm \pmdott p \pmimp q \pmdot \pmimp \pmdot \pmnot q \pmimp \pmnot p & \\
	& \boldsymbol{\pmast2\pmcdot17}. \quad \pmthm \pmdott \pmnot q \pmimp \pmnot p \pmdot \pmimp \pmdot p \pmimp q & 
\end{flalign*}

These four analogous propositions constitute the ``principle of transposition,'' referred to as ``Transp.'' They lead to the rule that in an implication the two sides may be interchanged by turning negative into positive and positive into negative. They are thus analogous to the algebraical rule that the two sides of an equation may be interchanged by changing the signs.
\begin{flalign*}
	& \boldsymbol{\pmast2\pmcdot04}. \quad \pmthm \pmdottt p \pmdot \pmimp \pmdot q \pmimp r \pmdott \pmimp \pmdott q \pmdot \pmimp \pmdot p \pmimp r &
\end{flalign*}

This is called the ``commutative principle'' and referred to as ``Comm.'' It states that, if $r$ follows from $q$ provided $p$ is true, then $r$ follows from $p$ provided $q$ is true.
\begin{flalign*}
	& \boldsymbol{\pmast2\pmcdot05}. \quad \pmthm \pmdottt q \pmimp r \pmdot \pmimp \pmdott p \pmimp q \pmdot \pmimp \pmdot p \pmimp r & \\
	& \boldsymbol{\pmast2\pmcdot06}. \quad \pmthm \pmdottt p \pmimp q \pmdot \pmimp \pmdott q \pmimp r \pmdot \pmimp \pmdot p \pmimp r &
\end{flalign*}

These two propositions are the source of the syllogism in Barbara (as will be shown later) and are therefore called the ``principle of the syllogism'' (referred to as ``Syll''). The first states that, if $r$ follows from $q$, then if $q$ follows from $p$, $r$ follows from $p$. The second states the same thing with the premisses interchanged.
\begin{flalign*}
	& \boldsymbol{\pmast2\pmcdot08}. \quad \pmthm \pmdot p \pmimp p &
\end{flalign*}

\textit{I.e.}\ any proposition implies itself. This is called the ``principle of identity'' and referred to as ``Id.'' It is not the same as the ``law of identity'' (``$x$ is identical with $x$''), but the law of identity is inferred from it (cf. $\pmast13\pmcdot15$).
\begin{flalign*}
	& \boldsymbol{\pmast2\pmcdot21}. \quad \pmthm \pmdott \pmnot p \pmdot \pmimp \pmdot p \pmimp q &
\end{flalign*}

\textit{I.e.}\ a false proposition implies any proposition.

The later propositions of the present number are mostly subsumed under propositions in $\pmast3$ or $\pmast4$, which give the same results in more compendious forms. We now proceed to formal deductions. 
\pmfd

\pageFsp{104}{100} \begin{flalign*} %2.01
	& \boldsymbol{\pmast2\pmcdot01}. \quad \pmthm \pmdott p \pmimp \pmnot p \pmdot \pmimp \pmdot \pmnot p &
\end{flalign*}

This proposition states that, if $p$ implies its own falsehood, then $p$ is false. It is called the ``principle of the \textit{reductio ad absurdum},'' and will be referred to as ``Abs.''\footnote{There is an interesting historical article on this principle by Vailati, ``A proposito d'un passo del Teeteto e di una dimostrazione di Euclide,'' \textit{Rivista di Filosofia e scienze affine}, 1904.} The proof is as follows (where ``\textit{Dem}.'' is short for ``demonstration''):
\\ \\ 
\pmdemi
\begin{flalign*} %2.01
	&& & \pmSub{\text{Taut}}{\pmnot p}{p}& &\pmthm \pmdott \pmnot p \pmor \pmnot p \pmdot \pmimp \pmdot \pmnot p& && (1)  \\
	&& & [(1)\pmdot(\pmast1\pmcdot01)]& &\pmthm \pmdott p \pmimp \pmnot p \pmdot \pmimp \pmdot \pmnot p& && 
\end{flalign*}
\begin{flalign*} %2.02
	& \boldsymbol{\pmast2\pmcdot02}. \quad \pmthm \pmdott q \pmdot \pmimp \pmdot p \pmimp q & && && && 
\end{flalign*}
\pmdemi
\begin{flalign*} %2.02
	&& & \pmSub{\text{Add}}{\pmnot p}{p} & &\pmthm \pmdott q \pmdot \pmimp \pmdot \pmnot p \pmor q & && (1)  \\
	&& & [(1)\pmdot(\pmast1\pmcdot01)]& &\pmthm \pmdott q \pmdot \pmimp \pmdot p \pmimp q & &&
\end{flalign*}
\begin{flalign*} %2.03
	& \boldsymbol{\pmast2\pmcdot03}. \quad \pmthm \pmdott p \pmimp \pmnot q \pmdot \pmimp \pmdot q \pmimp \pmnot p & && && && 
\end{flalign*}
\pmdemi
\begin{flalign*} %2.03
	&& & \pmSubb{\text{Perm}}{\pmnot p}{p}{\pmnot q}{q} & &\pmthm \pmdott \pmnot p \pmor \pmnot q \pmdot \pmimp \pmdot \pmnot q \pmor \pmnot p & && (1)  \\
	&& & [(1)\pmdot(\pmast1\pmcdot01)]& &\pmthm \pmdott p \pmimp \pmnot q \pmdot \pmimp \pmdot q \pmimp \pmnot p & &&
\end{flalign*}
\begin{flalign*} %2.04
	& \boldsymbol{\pmast2\pmcdot04}. \quad \pmthm \pmdottt p \pmdot \pmimp \pmdot q \pmimp r \pmdott \pmimp \pmdott q \pmdot \pmimp \pmdot p \pmimp r & && && && 
\end{flalign*}
\pmdemi
\begin{flalign*} %2.04
	&& & \pmSubb{\text{Assoc}}{\pmnot p}{p}{\pmnot q}{q} & &\pmthm \pmdottt \pmnot p \pmor (\pmnot q \pmor r) \pmdot \pmimp \pmdot \pmnot q \pmor (\pmnot p \pmor r) & && (1) \\
	&& & [(1)\pmdot(\pmast1\pmcdot01)]& &\pmthm \pmdottt p \pmdot \pmimp \pmdot q \pmimp r \pmdott \pmimp \pmdott q \pmdot \pmimp \pmdot p \pmimp r & &&
\end{flalign*}
\begin{flalign*} %2.05
	& \boldsymbol{\pmast2\pmcdot05}. \quad \pmthm \pmdottt q \pmimp r \pmdot \pmimp \pmdott p \pmimp q \pmdot \pmimp \pmdot p \pmimp r & && && && 
\end{flalign*}
\pmdemi
\begin{flalign*} %2.05
	&& & \pmSub{\text{Sum}}{\pmnot p}{p} & &\pmthm \pmdottt q \pmimp r \pmdot \pmimp \pmdott \pmnot p \pmor q \pmdot \pmimp \pmdot \pmnot p \pmor r & && (1) \\
	&& & [(1)\pmdot(\pmast1\pmcdot01)]& &\pmthm \pmdottt q \pmimp r \pmdot \pmimp \pmdott p \pmimp q \pmdot \pmimp \pmdot p \pmimp r & &&
\end{flalign*}
\begin{flalign*} %2.06
	& \boldsymbol{\pmast2\pmcdot06}. \quad \pmthm \pmdottt p \pmimp q \pmdot \pmimp \pmdott q \pmimp r \pmdot \pmimp \pmdot p \pmimp r & && && && 
\end{flalign*}
\pmdemi
\begin{flalign*} %2.06
	&& & \pmSubbb{\text{Comm}}{q \pmimp r}{p}{p \pmimp q}{q}{p \pmimp r}{r} & &\pmthm \pmdotttt q \pmimp r \pmdot \pmimp \pmdott p \pmimp q \pmdot \pmimp \pmdot p \pmimp r  \pmdottt & && \\
	&& && & \quad \pmimp \pmdottt p \pmimp q \pmdot \pmimp \pmdott q \pmimp r \pmdot \pmimp \pmdot p \pmimp r & && (1) \\
	&& & [\pmast2\pmcdot05] & &\pmthm \pmdottt q \pmimp r \pmdot \pmimp \pmdott p \pmimp q \pmdot \pmimp \pmdot p \pmimp r & && (2) \\
	&& & [(1)\pmdot(2)\pmdot\pmast1\pmcdot11]& &\pmthm \pmdottt p \pmimp q \pmdot \pmimp \pmdott q \pmimp r \pmdot \pmimp \pmdot p \pmimp r & &&
\end{flalign*}
In the last line of this proof, ``$(1)\pmdot(2)\pmdot\pmast1\pmcdot11$" means that we are inferring in accordance with $\pmast1\pmcdot11$, having before us a proposition, namely $p \pmimp q \pmdot \pmimp \pmdott q \pmimp r \pmdot \pmimp \pmdot p \pmimp r$, which, by (1), is implied by $q \pmimp r \pmdot \pmimp \pmdott p \pmimp q \pmdot \pmimp \pmdot p \pmimp r$, which, by (2), is true. In general, in such cases, we shall omit the reference to $\pmast1\pmcdot11$. 

\pageFsp{105}{101} The above two propositions will both be referred to as the ``principle of the syllogism" (shortened to ``Syll"), because, as will appear later, the syllogism in Barbara is derived from them. 
\begin{flalign*} %2.07
	& \boldsymbol{\pmast2\pmcdot07}. \quad \pmthm \pmdott p \pmdot \pmimp \pmdot p \pmor p \quad \pmSub{\pmast1\pmcdot3}{p}{q} & 
\end{flalign*}
Here we put nothing beyond ``$\pmast1\pmcdot3\dfrac{p}{q}$," because the proposition to be proved is what $\pmast1\pmcdot3$ becomes when $p$ is written in place of $q$. 

\begin{flalign*} %2.08
	& \boldsymbol{\pmast2\pmcdot08}. \quad \pmthm \pmdot p \pmimp p & 
\end{flalign*}
\pmdemi
\begin{flalign*} %2.08
	&& & \pmSubb{\pmast2\pmcdot05}{p \pmor p}{q}{p}{r} & &\pmthm \pmdotttt p \pmor p \pmdot \pmimp \pmdot p \pmdott \pmimp \pmdottt p \pmdot \pmimp \pmdot p \pmor p \pmdott \pmimp \pmdot p \pmimp p & && (1) \\
	&& & [\text{Taut}]& &\pmthm \pmdott p \pmor p \pmdot \pmimp \pmdot p  & && (2) \\
	&& & [(1)\pmdot(2)\pmdot\pmast1\pmcdot11]& &\pmthm \pmdottt p \pmdot \pmimp \pmdot p \pmor p \pmdott \pmimp \pmdot p \pmimp p  & && (3) \\
	&& & [\pmast2\pmcdot07]& &\pmthm \pmdott p \pmdot \pmimp \pmdot p \pmor p & && (4) \\
	&& & [(3)\pmdot(4)\pmdot\pmast1\pmcdot11]& & \pmthm \pmdot p \pmimp p  & && 
\end{flalign*}
\begin{flalign*} %2.1
	& \boldsymbol{\pmast2\pmcdot1}. \quad \pmthm \pmdot \pmnot p \pmor p \quad [\pmast2\pmcdot08\pmdot(\pmast1\pmcdot01)] & 
\end{flalign*}
\begin{flalign*} %2.11
	& \boldsymbol{\pmast2\pmcdot11}. \quad \pmthm \pmdot p \pmor \pmnot p & && && && 
\end{flalign*}
\pmdemi
\begin{flalign*} %2.11
	&& & \pmSubb{\text{Perm}}{\pmnot p}{p}{p}{q} & &\pmthm \pmdott \pmnot p \pmor p \pmdot \pmimp \pmdot p \pmor \pmnot p & && (1) \\ && &[(1)\pmdot\pmast2\pmcdot1\pmdot\pmast1\pmcdot11]& &\pmthm \pmdot p \pmor \pmnot p  & &&
\end{flalign*}
This is the law of excluded middle.
\begin{flalign*} %2.12
	& \boldsymbol{\pmast2\pmcdot12}. \quad \pmthm \pmdot p \pmimp \pmnot (\pmnot p) & && && && 
\end{flalign*}
\pmdemi
\begin{flalign*} %2.12
	&& & \pmSub{\pmast2\pmcdot11}{\pmnot p}{p} & &\pmthm \pmdot \pmnot p \pmor \pmnot (\pmnot p) & && (1) \\ && &[(1)\pmdot(\pmast1\pmcdot01)]& &\pmthm \pmdot p \pmimp \pmnot (\pmnot p)   & &&
\end{flalign*}
\pagef{106} \begin{flalign*} %2.13
	& \boldsymbol{\pmast2\pmcdot13}. \quad \pmthm \pmdot p \pmor \pmnot \{\pmnot (\pmnot p)\} & && && && 
\end{flalign*}

This proposition is a lemma for $\pmast2\pmcdot14$, which, with $\pmast2\pmcdot12$, constitutes the principle of double negation.
\\ \\ 
\pmdemi
\begin{flalign*} %2.13
	&& & \pmSubb{\text{Sum}}{\pmnot p}{q}{\pmnot\{\pmnot(\pmnot p)\}}{r} & &\pmthm \pmdottt \pmnot p \pmdot \pmimp \pmdot \pmnot\{\pmnot(\pmnot p)\} \pmdot \pmimp \pmdott & \\
	&& && & \qquad \quad p \pmor \pmnot p \pmdot \pmimp \pmdot p \pmor \pmnot\{\pmnot(\pmnot p)\} & (1) \\
	&& & \pmSub{\pmast2\pmcdot12}{\pmnot p}{p} & &\pmthm \pmdott \pmnot p \pmdot \pmimp \pmdot \pmnot\{\pmnot(\pmnot p)\}  & (2) \\
	&& & [(1)\pmdot(2)\pmdot\pmast1\pmcdot11]& &\pmthm \pmdott p \pmor \pmnot p \pmdot \pmimp \pmdot p \pmor \pmnot\{\pmnot(\pmnot p)\} & (3) \\
	&& & [(3)\pmdot\pmast2\pmcdot11\pmdot\pmdot\pmast1\pmcdot11]& & \pmthm \pmdot p \pmor \pmnot\{\pmnot(\pmnot p)\}  &  
\end{flalign*}

\pagesp{102} \begin{flalign*} %2.14
	& \boldsymbol{\pmast2\pmcdot14}. \quad \pmthm \pmdot \pmnot (\pmnot p) \pmimp p &  
\end{flalign*}
\pmdemi
\begin{flalign*} %2.14
	&& & \pmSub{\text{Perm}}{\pmnot\{\pmnot(\pmnot p)\}}{q} & &\pmthm \pmdott p \pmor \pmnot \{\pmnot (\pmnot p)\} \pmdot \pmimp \pmdot \pmnot\{\pmnot(\pmnot p)\} \pmor p & (1) \\ 
	&& &[(1)\pmdot\pmast2\pmcdot13\pmdot\pmast1\pmcdot11]& &\pmthm \pmdot \pmnot\{\pmnot(\pmnot p)\} \pmor p & (2) \\
	&& & [(2)\pmdot(\pmast1\pmcdot01)]& & \pmthm \pmdot \pmnot (\pmnot p) \pmimp p & 
\end{flalign*}
\begin{flalign*} %2.15
	& \boldsymbol{\pmast2\pmcdot15}. \quad \pmthm \pmdott \pmnot p \pmimp q \pmdot \pmimp \pmdot \pmnot q \pmimp p &  
\end{flalign*}
\pmdemi
\begin{flalign*} %2.15
	&\pmSubb{\pmast2\pmcdot05}{\pmnot p}{p}{\pmnot(\pmnot q)}{r} & &\pmthm \pmdottt q \pmimp \pmnot(\pmnot q) \pmdot \pmimp \pmdott \pmnot p \pmimp q \pmdot \pmimp \pmdot \pmnot p \pmimp \pmnot(\pmnot q) & (1) \\ 
	&\pmSub{\pmast2\pmcdot12}{q}{p} & &\pmthm \pmdot q \pmimp \pmnot(\pmnot q) & (2) \\ 
	&[(1)\pmdot(2)\pmdot\pmast1\pmcdot11] & &\pmthm \pmdott \pmnot p \pmimp q \pmdot \pmimp \pmdot \pmnot p \pmimp \pmnot(\pmnot q) & (3) \\
	&\pmSubb{\pmast2\pmcdot03}{\pmnot p}{p}{\pmnot q}{q} & &\pmthm \pmdott \pmnot p \pmimp \pmnot(\pmnot q) \pmdot \pmimp \pmdot \pmnot q \pmimp \pmnot(\pmnot p) & (4) \\ 
	&\pmSubbb{\pmast2\pmcdot05}{\pmnot q}{p}{\pmnot(\pmnot p)}{q}{p}{r} & &\pmthm \pmdottt \pmnot(\pmnot p) \pmimp p \pmdot \pmimp \pmdott \pmnot q \pmimp \pmnot(\pmnot p) \pmdot \pmimp \pmdot \pmnot q \pmimp p & (5) \\ 
	&[(5)\pmdot\pmast2\pmcdot14\pmdot\pmast1\pmcdot11] & & \pmthm \pmdott \pmnot q \pmimp \pmnot(\pmnot p) \pmdot \pmimp \pmdot \pmnot q \pmimp p & (6) \\
	&\multispan3{$\pmSubbb{\pmast2\pmcdot05}{\pmnot p \pmimp q}{p}{\pmnot p \pmimp \pmnot(\pmnot q)}{q}{\pmnot q \pmimp \pmnot(\pmnot p)}{r} \quad \pmthm \pmdotttt \pmnot p \pmimp \pmnot(\pmnot q) \pmdot \pmimp \pmdot \pmnot q \pmimp \pmnot (\pmnot p) \pmdott \pmimp \pmdottt$} &\\ 
	&& & \pmnot p \pmimp q \pmdot \pmimp \pmdot \pmnot p \pmimp \pmnot(\pmnot q) \pmdott \pmimp \pmdott \pmnot p \pmimp q \pmdot \pmimp \pmdot \pmnot q \pmimp \pmnot(\pmnot p) &  (7) \\ 
	&[(4)\pmdot(7)\pmdot\pmast1\pmcdot11]& & \pmthm \pmdottt \pmnot p \pmimp q \pmdot \pmimp \pmdot \pmnot p \pmimp \pmnot(\pmnot q) \pmdott \pmimp \pmdott \pmnot p \pmimp q \pmdot \pmimp \pmdot \pmnot q \pmimp \pmnot(\pmnot p) &  (8) \\
	&[(3)\pmdot(8)\pmdot\pmast1\pmcdot11]& & \pmthm \pmdott \pmnot p \pmimp q \pmdot \pmimp \pmdot \pmnot q \pmimp \pmnot(\pmnot p) &  (9) \\
	&\multispan3{$\pmSubbb{\pmast2\pmcdot05}{\pmnot p \pmimp q}{p}{\pmnot q \pmimp \pmnot(\pmnot p)}{q}{\pmnot q \pmimp p}{r} \quad \pmthm \pmdotttt \pmnot q \pmimp \pmnot(\pmnot p) \pmdot \pmimp \pmdot \pmnot q \pmimp p \pmdott \pmimp \pmdott$ \hfill} &  \\ 
	&& & \pmimp \pmdottt \pmnot p \pmimp q \pmdot \pmimp \pmdot \pmnot q \pmimp \pmnot(\pmnot p) \pmdott \pmimp \pmdott \pmnot p \pmimp q \pmdot \pmimp \pmdot \pmnot q \pmimp p & (10) \\ 
	&[(6)\pmdot(10)\pmdot\pmast1\pmcdot11]& & \pmthm \pmdottt \pmnot p \pmimp q \pmdot \pmimp \pmdot \pmnot q \pmimp \pmnot(\pmnot p) \pmdott \pmimp \pmdott \pmnot p \pmimp q \pmdot \pmimp \pmdot \pmnot q \pmimp p &  (11) \\
	&[(9)\pmdot(11)\pmdot\pmast1\pmcdot11]& & \pmthm \pmdott \pmnot p \pmimp q \pmdot \pmimp \pmdot \pmnot q \pmimp p & \\
\end{flalign*}  \pagef{107} 

\textit{Note on the proof of $\pmast2\pmcdot15$}. In the above proof, it will be seen that (3), (4), (6) are respectively of the forms $p_1 \pmimp p_2$, $p_2 \pmimp p_3$, $p_3 \pmimp p_4$, where $p_1 \pmimp p_4$ is the proposition to be proved. From $p_1 \pmimp p_2$, $p_2 \pmimp p_3$, $p_3 \pmimp p_4$ the proposition $p_1 \pmimp p_4$ results by repeated applications of $\pmast2\pmcdot05$ or $\pmast2\pmcdot06$ (both of which are called ``Syll"). It is tedious and unnecessary to repeat this process every time it is used; it will therefore be abbreviated into
\[
\text{``[Syll] } \pmthm \pmdot (a) \pmand (b) \pmand (c) \pmdot \pmimp \pmthm \pmdot (d)\text{,"}
\]
where $(a)$ is of the form $p_1 \pmimp p_2$, $(b)$ of the form $p_2 \pmimp p_3$, $(c)$ of the form $p_3 \pmimp p_4$, and $(d)$ of the form $p_1 \pmimp p_4$. The same abbreviation will be applied to a sorites of any length.

\pagesp{103} Also where we have ``$\pmthm \pmdot p_1$" and ``$\pmthm \pmdot p_1 \pmimp p_2$," and $p_2$ is the proposition to
be proved, it is convenient to write simply 
\begin{flalign*}
	&& \text{``}\:&\pmthm \pmdot p_1 \pmdot \pmimp & && \\
	&\text{[etc.]} & &\pmthm \pmdot p_2\text{,"}& &&
\end{flalign*}
where ``etc." will be a reference to the previous propositions in virtue of which the implication ``$\pmdot p_1 \pmimp p_2$" holds. This form embodies the use of $\pmast1\pmcdot11$ or $\pmast1\pmcdot1$, and makes many proofs at once shorter and easier to follow. It is used in the first two lines of the following proof.
\begin{flalign*} %2.16
	& \boldsymbol{\pmast2\pmcdot16}. \quad \pmthm \pmdott p \pmimp q \pmdot \pmimp \pmdot \pmnot q \pmimp \pmnot p & && && && 
\end{flalign*}
\pmdemi
\begin{flalign*} %2.16
	&& & [\pmast2\pmcdot12]& &\pmthm \pmdot q \pmimp \pmnot (\pmnot q) \pmdot \pmimp  & && \\
	&& & [\pmast2\pmcdot05]& &\pmthm \pmdott p \pmimp q \pmdot \pmimp \pmdot p \pmimp \pmnot(\pmnot q)  & && (1) \\
	&& & \pmSub{\pmast2\pmcdot03}{\pmnot q}{q} & &\pmthm \pmdott p \pmimp \pmnot(\pmnot q) \pmdot \pmimp \pmdot \pmnot q \pmimp \pmnot p & && (2) \\
	&& & [\text{Syll}]& &\pmthm \pmdot (1) \pmand (2) \pmdot \pmithm \pmdott p \pmimp q \pmdot \pmimp \pmdot \pmnot q \pmimp \pmnot p & && 
\end{flalign*}

\pagef{108} \textit{Note}. The proposition to be proved will be called ``\pmprop," and when a proof ends, like that of $\pmast2\pmcdot16$, by an implication between asserted propositions, of which the consequent is the proposition to be proved, we shall write ``$\pmthm \pmdot \text{etc.} \pmithm \pmdot \pmprop$". Thus ``$\pmithm \pmdot \pmprop$" ends a proof, and more or less corresponds to ``\begin{scriptsize}Q.E.D.\end{scriptsize}"
\begin{flalign*} %2.17
	& \boldsymbol{\pmast2\pmcdot17}. \quad \pmthm \pmdott \pmnot q \pmimp \pmnot p \pmdot \pmimp \pmdot p \pmimp q & && && && 
\end{flalign*}
\pmdemi
\begin{flalign*} %2.17
	&& & \pmSubb{\pmast2\pmcdot03}{\pmnot q}{p}{p}{q} & &\pmthm \pmdott \pmnot q \pmimp \pmnot p \pmdot \pmimp \pmdot p \pmimp \pmnot(\pmnot q) & && (1) \\
	&& & [\pmast2\pmcdot14]& &\pmthm \pmdot \pmnot (\pmnot q) \pmimp q \pmdott \pmimp & && \\
	&& & [\pmast2\pmcdot05]& &\pmthm \pmdott p \pmimp \pmnot(\pmnot q) \pmdot \pmimp \pmdot p \pmimp q  & && (2) \\
	&& & [\text{Syll}]& &\pmthm \pmdot (1) \pmand (2) \pmdot \pmithm \pmprop & && 
\end{flalign*}

$\pmast2\pmcdot15$, $\pmast2\pmcdot16$ and $\pmast2\pmcdot17$ are forms of the principle of transposition, and will be all referred to as ``Transp."
\begin{flalign*} %2.18
	& \boldsymbol{\pmast2\pmcdot18}. \quad \pmthm \pmdott \pmnot p \pmimp p \pmdot \pmimp \pmdot p & && && && 
\end{flalign*}
\pmdemi
\begin{flalign*} %2.18
	&& & [\pmast2\pmcdot12]& &\pmthm \pmdot p \pmimp \pmnot (\pmnot p) \pmdot \pmimp & && \\
	&& & [\pmast2\pmcdot05]& &\pmthm \pmdot \pmnot p \pmimp p \pmdot \pmimp \pmdot \pmnot p \pmimp \pmnot(\pmnot p)  & && (1) \\
	&& & \pmSub{\pmast2\pmcdot01}{\pmnot p}{p}& &\pmthm \pmdott \pmnot p \pmimp \pmnot(\pmnot p) \pmdot \pmimp \pmdot \pmnot(\pmnot p) & && (2) \\
	&& & [\text{Syll}]& &\pmthm \pmdot (1) \pmand (2) \pmdot \pmithm \pmdott \pmnot p \pmimp p \pmdot \pmimp \pmdot \pmnot(\pmnot p)  & && (3) \\
	&& & [\pmast2\pmcdot14]& &\pmthm \pmdot \pmnot (\pmnot p) \pmimp p & && (4) \\
	&& & [\text{Syll}]& &\pmthm \pmdot (3) \pmand (4) \pmdot \pmithm \pmdot \pmprop  & && \\
\end{flalign*}

This is the complement of the principle of the \textit{reductio ad absurdum}. It \pagesp{104} states that a proposition which follows from the hypothesis of its own falsehood is true.
\begin{flalign*} %2.2
	& \boldsymbol{\pmast2\pmcdot2}. \quad \pmthm \pmdott p \pmdot \pmimp \pmdot p \pmor q & 
\end{flalign*}
\pmdemi
\begin{flalign*} %2.2
	&& & \pmthm \pmdot \text{Add} \pmdot \pmithm \pmdott p \pmimp q \pmor p & (1) \\
	&& & [\text{Perm}]\; \pmthm \pmdott q \pmor p \pmdot \pmimp \pmdot p \pmor q & (2) \\
	&& & [\text{Syll}]\;  \pmthm \pmdot (1) \pmand (2) \pmdot \pmithm \pmdot \pmprop  & \\
\end{flalign*}
\begin{flalign*} %2.21
	& \boldsymbol{\pmast2\pmcdot21}. \quad \pmthm \pmdott \pmnot p \pmdot \pmimp \pmdot p \pmimp q \quad \pmSub{\pmast2\pmcdot2}{\pmnot p}{p} & 
\end{flalign*}

The above two propositions are very frequently used.
\begin{flalign*} %2.24
	& \boldsymbol{\pmast2\pmcdot24}. \quad \pmthm \pmdott p \pmdot \pmimp \pmdot \pmnot p \pmimp q \quad [\pmast2\pmcdot21 \pmdot \text{Comm}] & 
\end{flalign*}
\pagef{109} \begin{flalign*} %2.25
	& \boldsymbol{\pmast2\pmcdot25}. \quad \pmthm \pmdottt p \pmdott \pmor \pmdott p \pmor q \pmdot \pmimp \pmdot q & 
\end{flalign*}
\pmdemi
\begin{flalign*} %2.25
	&& & \pmthm \pmdot \pmast2\pmcdot1 \pmdot \pmithm \pmdott \pmnot(p \pmor q) \pmdot \pmor \pmdot (p \pmor q) \pmdott & \\
	&& & [\text{Assoc}]\:\: \pmithm \pmdott p \pmdot \pmor \pmdot \{\pmnot(p \pmor q) \pmdot \pmor \pmdot q\} \pmdott \pmithm \pmdot \pmprop  & 
\end{flalign*}
\begin{flalign*} %2.26
	& \boldsymbol{\pmast2\pmcdot26}. \quad \pmthm \pmdottt \pmnot p \pmdott \pmor \pmdott p \pmimp q \pmdot \pmimp \pmdot q \quad \pmSub{\pmast2\pmcdot25}{\pmnot p}{p} & 
\end{flalign*}
\begin{flalign*} %2.27
	& \boldsymbol{\pmast2\pmcdot27}. \quad \pmthm \pmdottt p \pmdot \pmimp \pmdott p \pmimp q \pmdot \pmimp \pmdot q \quad [\pmast2\pmcdot26] & 
\end{flalign*}
\begin{flalign*} %2.3
	& \boldsymbol{\pmast2\pmcdot3}. \quad \pmthm \pmdott p \pmor (q \pmor r) \pmdot \pmimp \pmdot p \pmor (r \pmor q)  & 
\end{flalign*}
\pmdemi
\begin{flalign*} %2.3
	&& & \pmSubb{\text{Perm}}{q}{p}{r}{q} \;\:\qquad\qquad \pmthm \pmdott (q \pmor r) \pmdot \pmimp \pmdot (r \pmor q) \pmdott & \\
	&& & \pmSubb{\text{Sum}}{q \pmor r}{q}{r \pmor r}{r}\:\: \pmithm \pmdott p \pmor (q \pmor r) \pmdot \pmimp \pmdot p \pmor (r \pmor q) & 
\end{flalign*}
\begin{flalign*} %2.31
	& \boldsymbol{\pmast2\pmcdot31}. \quad \pmthm \pmdott p \pmor (q \pmor r) \pmdot \pmimp \pmdot (p \pmor q) \pmor r & 
\end{flalign*}

This proposition and $\pmast2\pmcdot32$ together constitute the associative law for logical addition of propositions. In the proof, the following abbreviation (constantly used hereafter) will be employed\footnote{This abbreviation applies to the same type of cases as those concerned in the note to $\pmast2\pmcdot15$, but is often more convenient than the abbreviation explained in that note.}: When we have a series of propositions of the form $a \pmimp b$, $b \pmimp c$, $c \pmimp d$, all asserted, and ``$a \pmimp d$" is the proposition to be proved, the proof io full is as follows:
\begin{flalign*} %2.31
	&[\text{Syll}] & &\pmthm \pmdottt a \pmimp b \pmdot \pmimp \pmdott b \pmimp c \pmdot \pmimp \pmdot a \pmimp c & (1) \\ 
	& & &\pmthm \pmdott a \pmdot \pmimp \pmdot b & (2) \\ 
	&[(1)\pmdot(2)\pmdot\pmast1\pmcdot11] & &\pmthm \pmdott b \pmimp c \pmdot \pmimp \pmdot a \pmimp c  & (3) \\
	& & &\pmthm \pmdott b \pmdot \pmimp \pmdot c & (4) \\
	&[(3)\pmdot(4)\pmdot\pmast1\pmcdot11] & &\pmthm \pmdott a \pmdot \pmimp \pmdot c  &  (5) \\ 
	&[\text{Syll}] & &\pmthm \pmdottt a \pmimp c \pmdot \pmimp \pmdott c \pmimp d \pmdot \pmimp \pmdot a \pmimp d & (6) \\ 
	&[(5)\pmdot(6)\pmdot\pmast1\pmcdot11] & &\pmthm \pmdott c \pmimp d \pmdot \pmimp \pmdot a \pmimp d  & (7) \\
	& & &\pmthm \pmdott c \pmdot \pmimp \pmdot d & (8) \\
	&[(7)\pmdot(8)\pmdot\pmast1\pmcdot11] & &\pmthm \pmdott a \pmdot \pmimp \pmdot d  & \\
\end{flalign*}

\pagesp{105} It is tedious to write out this process in full; we therefore write simply
\begin{flalign*}
	&& &\pmthm\: \pmdott a \pmdot \pmimp \pmdot b \pmdot & \\
	&& &[\text{etc.}]\; \pmimp \pmdot c \pmdot & \\
	&& &[\text{etc.}]\; \pmimp \pmdot d \pmdott \pmithm \pmdot \pmprop, &
\end{flalign*}
where ``$a \pmimp d$" is the proposition to be proved. We indicate on the left by references in square brackets the propositions in virtue of which the successive implications hold. We put one dot (not two) after ``$b$," to show \pagef{110} that it is $b$, not ``$a \pmimp b$," that implies $c$. But we put two dots after $d$, to show that now the whole proposition ``$a \pmimp d$" is concerned. If ``$a \pmimp d$" is not the proposition to be proved, but is to be used subsequently in the proof, we put
\begin{flalign*}
	&& &\pmthm\: \pmdott a \pmdot \pmimp \pmdot b \pmdot & \\
	&& &[\text{etc.}]\; \pmimp \pmdot c \pmdot & \\
	&& &[\text{etc.}]\; \pmimp \pmdot d \pmdott \pmithm \pmdot \pmprop & (1),
\end{flalign*}
and then ``(1)" means ``$a \pmimp d$." The proof of $\pmast2\pmcdot31$ is as follows:
\\ \\
\pmdemi
\begin{flalign*} %2.31
	&& && && && & \multispan7{$[\pmast2\pmcdot3]\:\pmthm \pmdott p \pmor (q \pmor r) \pmdot \pmimp \pmdot p \pmor (r \pmor q) \pmdot $ \hfill} & && && && \\
	&& && && && & \pmSubb{\text{Assoc}}{r}{q}{q}{r} & & \pmimp \pmdot r \pmor (p \pmor q) \pmdot & && && \\
	&& && && && & \pmSubb{\text{Perm}}{q}{p}{r}{q} & & \pmimp \pmdott (q \pmor r) \pmdot \pmimp \pmdot (r \pmor q) \pmdott \pmithm \pmdot \pmprop & && &&
\end{flalign*}
\begin{flalign*} %2.32
	& \boldsymbol{\pmast2\pmcdot32}. \quad \pmthm \pmdott (p \pmor q) \pmor r \pmdot \pmimp \pmdot p \pmor (q \pmor r) & 
\end{flalign*}
\pmdemi
\begin{flalign*} %2.32
	&& &\multispan4{$\pmSubb{\text{Perm}}{p \pmor q}{p}{r}{q}\:\pmthm \pmdott (p \pmor q) \pmor r \pmdot \pmimp \pmdot r \pmor (p \pmor q)$ \hfill} & && && \\
	&& &  \pmSubbb{\text{Assoc}}{r}{p}{p}{q}{q}{r} & & \quad\pmimp \pmdot p \pmor (r \pmor q) \pmdot & && \\
	&& & [\pmast2\pmcdot3] & & \quad\pmimp \pmdot p \pmor (q \pmor r) \pmdott \pmithm \pmdot \pmprop & &&
\end{flalign*}
\begin{flalign*} %2.33
	& \boldsymbol{\pmast2\pmcdot33}. \quad p \pmor q \pmor r \pmdot \pmiddf \pmdot (p \pmor q) \pmor r \pmdf & 
\end{flalign*}

This definition serves only for the avoidance of brackets.

\begin{flalign*} %2.36
	& \boldsymbol{\pmast2\pmcdot36}. \quad \pmthm \pmdott q \pmimp r \pmdot \pmimp \pmdott q \pmor p \pmdot \pmimp \pmdot r \pmor p & 
\end{flalign*}
\pmdemi
\begin{flalign*} %2.36
	&& & [\text{Perm}] & \pmthm &\pmdott p \pmor r \pmdot \pmimp \pmdot r \pmor p \pmdott & \\
	&& &  \pmSubbb{\text{Syll}}{p \pmor q}{p}{p \pmor r}{q}{r \pmor p}{r} & \pmithm & \pmdottt p \pmor q \pmdot \pmimp \pmdot p \pmor r \pmdott \pmimp \pmdott p \pmor q \pmdot \pmimp \pmdot r \pmor p & (1) \\
	&& & [\text{Sum}] & \pmthm & \pmdottt q \pmimp r \pmdot \pmimp \pmdott p \pmor q \pmdot \pmimp \pmdot p \pmor r & (2) \\
	&& & \pmthm \pmdot(1)\pmdot(2)\pmdot\text{Syll}\pmdot\pmithm\pmdot \pmprop & &&
\end{flalign*}
\begin{flalign*} %2.37
	& \boldsymbol{\pmast2\pmcdot37}. \quad \pmthm \pmdottt q \pmimp r \pmdot \pmimp \pmdott q \pmor p \pmdot \pmimp \pmdot p \pmor r \quad [\text{Syll} \pmdot \text{Perm} \pmdot \text{Sum}] & 
\end{flalign*}
\begin{flalign*} %2.38
	& \boldsymbol{\pmast2\pmcdot38}. \quad \pmthm \pmdottt q \pmimp r \pmdot \pmimp \pmdott q \pmor p \pmdot \pmimp \pmdot r \pmor p \quad [\text{Syll} \pmdot \text{Perm} \pmdot \text{Sum}] & 
\end{flalign*}

\pagesp{106} The proofs of $\pmast2\pmcdot37\pmcdot38$ are exactly analogous to that of $\pmast2\pmcdot36$. (We use ``$\pmast2\pmcdot37\pmcdot38$" as an abbreviation for ``$\pmast2\pmcdot37$ and $\pmast2\pmcdot38$." Such abbreviations will be used throughout.)

\pagef{111} The use of a general principle of deduction, such as either form of ``Syll," in a proof, is different from the use of the particular premisses to which the principle of deduction is applied. The principle of deduction gives the general rule according to which the inference is made, but is not itself a premiss in the inference. If we treated it as a premiss, we should need either it or some other general rule to enable us to infer the desired conclusion, and thus we should gradually acquire an increasing accumulation of premisses without ever being able to make any inference. Thus when a general rule is adduced in drawing an inference, as when we write ``$[\text{Syll}]\: \pmthm \pmdot (1) \pmdot (2) \pmdot \pmithm \:\pmdot \pmprop$," the mention of ``Syll" is only required in order to remind the reader how the inference is drawn. 

The rule of inference may, however, also occur as one of the ordinary premisses, that is to say, in the case of ``Syll" for example, the proposition ``$p \pmimp q \pmdot \pmimp \pmdott q \pmimp r \pmdot \pmimp \pmdot p \pmimp r$" may be one of those to which our rules of deduction are applied, and it is then an ordinary premiss. The distinction between the two uses of principles of deduction is of some philosophical importance, and in the above proofs we have indicated it by putting the rule of inference in square brackets. It is, however, practically inconvenient to continue to distinguish in the manner of the reference. We shall therefore henceforth both adduce ordinary premisses in square brackets where convenient, and adduce rules of inference, along with other propositions, in
asserted premisses, \textit{i.e.}\ we shall write \textit{e.g.}\
\begin{flalign*}
	&& &\text{``} \pmthm \pmdot (1) \pmdot (2) \pmdot \text{Syll} \pmdot \pmithm\: \pmdot \pmprop\text{"} & \\
	&\text{rather than} & &\text{``}[\text{Syll}] \: \pmthm \pmdot (1) \pmdot (2) \pmdot\pmdot \pmithm\: \pmdot \pmprop\text{"} & 
\end{flalign*}
\begin{flalign*} %2.4
	& \boldsymbol{\pmast2\pmcdot4}. \quad \pmthm \pmdottt p \pmdot \pmor \pmdot p \pmor q \pmdott \pmimp \pmdot p \pmor q & 
\end{flalign*}
\pmdemi
\begin{flalign*} %2.4
	&& & \pmthm \pmdottt \pmast2\pmcdot31 \pmdot \pmithm \pmdottt & p \pmdot \pmor \pmdot p \pmor q \pmdott\; & \pmimp \pmdot p \pmor p \pmdot \pmor \pmdot q \pmdott & \\
	&& & [\text{Sum}\pmdot\pmast2\pmcdot38] & & \pmimp \pmdott p\pmor q \pmdottt \pmithm\pmdot \pmprop & 
\end{flalign*}
\begin{flalign*} %2.41
	& \boldsymbol{\pmast2\pmcdot41}. \quad \pmthm \pmdottt q \pmdot \pmor \pmdot p \pmor q \pmdott \pmimp \pmdot p \pmor q  & 
\end{flalign*}
\pmdemi
\begin{flalign*} %2.41
	&& &  \pmSubbb{\text{Assoc}}{q}{p}{p}{q}{q}{r} & \pmthm \pmdottt q \pmdot \pmor \pmdot p \pmor q \pmdott \; &\pmimp \pmdott p \pmdot \pmor \pmdot q \pmor q \pmdott & \\
	&& & [\text{Taut}\pmdot\text{Sum}] && \pmimp \pmdott p \pmor q \pmdottt \pmithm\pmdot \pmprop &
\end{flalign*}
\begin{flalign*} %2.42
	& \boldsymbol{\pmast2\pmcdot42}. \quad \pmthm \pmdottt \pmnot p \pmdot \pmor \pmdot p \pmimp q \pmdott \pmimp \pmdot p \pmimp q \quad \pmSub{\pmast2\pmcdot4}{\pmnot p}{p}  & 
\end{flalign*}
\begin{flalign*} %2.43
	& \boldsymbol{\pmast2\pmcdot43}. \quad \pmthm \pmdottt p \pmdot \pmimp \pmdot p \pmimp q \pmdott \pmimp \pmdot p \pmimp q \quad [\pmast2\pmcdot42] & 
\end{flalign*}
\begin{flalign*} %2.45
	& \boldsymbol{\pmast2\pmcdot45}. \quad \pmthm \pmdott \pmnot(p \pmor q) \pmdot \pmimp \pmdot \pmnot p \quad [\pmast2\pmcdot2 \pmdot \text{Transp}] & 
\end{flalign*}
\begin{flalign*} %2.46
	& \boldsymbol{\pmast2\pmcdot46}. \quad \pmthm \pmdott \pmnot(p \pmor q) \pmdot \pmimp \pmdot \pmnot q \quad [\pmast1\pmcdot3 \pmdot \text{Transp}] & 
\end{flalign*}
\pageFsp{112}{107} 
\begin{flalign*} %2.47
	& \boldsymbol{\pmast2\pmcdot47}. \quad \pmthm \pmdott \pmnot(p \pmor q) \pmdot \pmimp \pmdot \pmnot p \pmor q \quad \pmbr{\pmast2\pmcdot45\pmdot\pmSUb{\pmast2\pmcdot2}{\pmnot p}{p}\pmdot\text{Syll}} & 
\end{flalign*}
\begin{flalign*} %2.48
	& \boldsymbol{\pmast2\pmcdot48}. \quad \pmthm \pmdott \pmnot(p \pmor q) \pmdot \pmimp \pmdot p \pmor \pmnot q \quad \pmbr{\pmast2\pmcdot46\pmdot\pmSUb{\pmast1\pmcdot3}{\pmnot q}{q}\pmdot\text{Syll}} & 
\end{flalign*}
\begin{flalign*} %2.49
	& \boldsymbol{\pmast2\pmcdot49}. \quad \pmthm \pmdott \pmnot(p \pmor q) \pmdot \pmimp \pmdot \pmnot p \pmor \pmnot q \quad \pmbr{\pmast2\pmcdot45\pmdot\pmSUbb{\pmast2\pmcdot2}{\pmnot p}{p}{\pmnot q}{q}\pmdot\text{Syll}} & 
\end{flalign*}
\begin{flalign*} %2.5
	& \boldsymbol{\pmast2\pmcdot5}. \quad \pmthm \pmdott \pmnot(p \pmimp q) \pmdot \pmimp \pmdot \pmnot p \pmimp q \quad \pmSub{\pmast2\pmcdot47}{\pmnot p}{p} & 
\end{flalign*}
\begin{flalign*} %2.51
	& \boldsymbol{\pmast2\pmcdot51}. \quad \pmthm \pmdott \pmnot(p \pmimp q) \pmdot \pmimp \pmdot p \pmimp \pmnot q \quad \pmSub{\pmast2\pmcdot48}{\pmnot p}{p} & 
\end{flalign*}
\begin{flalign*} %2.52
	& \boldsymbol{\pmast2\pmcdot52}. \quad \pmthm \pmdott \pmnot(p \pmimp q) \pmdot \pmimp \pmdot \pmnot p \pmimp \pmnot q \quad \pmSub{\pmast2\pmcdot49}{\pmnot p}{p} & 
\end{flalign*}
\begin{flalign*} %2.521
	& \boldsymbol{\pmast2\pmcdot521}. \quad \pmthm \pmdott \pmnot(p \pmimp q) \pmdot \pmimp \pmdot q \pmimp p \quad [\pmast2\pmcdot52\pmcdot17] & 
\end{flalign*}
\begin{flalign*} %2.53
	& \boldsymbol{\pmast2\pmcdot53}. \quad \pmthm \pmdott p \pmor q \pmdot \pmimp \pmdot \pmnot p \pmimp q & 
\end{flalign*}
\pmdemi
\begin{flalign*} %2.53
	\pmthm \pmdot \pmast2\pmcdot12\pmcdot38\pmdot\pmithm\pmdott p \pmor q \pmdot \pmimp\pmdot\pmnot(\pmnot p)\pmor q \pmdott \pmithm \pmdot \pmprop
\end{flalign*}
\begin{flalign*} %2.54
	& \boldsymbol{\pmast2\pmcdot54}. \quad \pmthm \pmdott \pmnot p \pmimp q \pmdot \pmimp \pmdot p \pmor q \quad [\pmast2\pmcdot14\pmcdot38] & 
\end{flalign*}
\begin{flalign*} %2.55
	& \boldsymbol{\pmast2\pmcdot55}. \quad \pmthm \pmdottt \pmnot p \pmdot \pmimp\pmdott p \pmor q \pmdot \pmimp \pmdot q \quad [\pmast2\pmcdot53\pmdot\text{Comm}] & 
\end{flalign*}
\begin{flalign*} %2.56
	& \boldsymbol{\pmast2\pmcdot56}. \quad \pmthm \pmdottt \pmnot q \pmdot \pmimp\pmdott p \pmor q \pmdot \pmimp \pmdot p \quad \pmbr{\pmSUbb{\pmast2\pmcdot55}{q}{p}{p}{q}\pmdot\text{Perm}} & 
\end{flalign*}
\begin{flalign*} %2.6
	& \boldsymbol{\pmast2\pmcdot6}. \quad \pmthm \pmdottt \pmnot p \pmimp q \pmdot \pmimp \pmdott p \pmimp q \pmdot \pmimp \pmdot q & 
\end{flalign*}
\pmdemi
\begin{flalign*}%2.6
	&& & [\pmast2\pmcdot38] & &\pmthm \pmdottt \pmnot p \pmimp q \pmdot \pmimp \pmdott \pmnot p \pmor q \pmdot \pmimp \pmdot q \pmor q & (1) \\
	&& & [\text{Taut}\pmdot\text{Syll}] & &\pmthm \pmdottt \pmnot p \pmor q \pmdot \pmimp \pmdot q \pmor q \pmdott \pmimp \pmdott \pmnot p \pmor q \pmdot \pmimp \pmdot q & (2) \\
	&& & \multispan4{$(1)\pmdot(2)\pmdot\text{Syll}\pmdot \pmithm \pmdottt \pmnot p \pmimp q \pmdot \pmimp \pmdott \pmnot p \pmor q \pmdot \pmimp \pmdot q \pmdottt \pmithm \pmdot \pmprop$\hfill} & &&
\end{flalign*}
\begin{flalign*} %2.61
	& \boldsymbol{\pmast2\pmcdot61}. \quad \pmthm \pmdottt p \pmimp q \pmdot \pmimp \pmdott \pmnot p \pmimp q \pmdot \pmimp \pmdot q \quad [\pmast2\pmcdot6\pmdot\text{Comm}] & 
\end{flalign*}
\begin{flalign*} %2.62
	& \boldsymbol{\pmast2\pmcdot62}. \quad \pmthm \pmdottt p \pmor q \pmdot \pmimp \pmdott p \pmimp q \pmdot \pmimp \pmdot q \quad [\pmast2\pmcdot53\pmcdot6\pmdot\text{Syll}] & 
\end{flalign*}
\begin{flalign*} %2.621
	& \boldsymbol{\pmast2\pmcdot621}. \quad \pmthm \pmdottt p \pmimp q \pmdot \pmimp \pmdott p \pmor q \pmdot \pmimp \pmdot q \quad [\pmast2\pmcdot62\pmdot\text{Comm}] & 
\end{flalign*}
\begin{flalign*} %2.63
	& \boldsymbol{\pmast2\pmcdot63}. \quad \pmthm \pmdottt p \pmor q \pmdot \pmimp \pmdott \pmnot p \pmor q \pmdot \pmimp \pmdot q \quad [\pmast2\pmcdot62] & 
\end{flalign*}
\begin{flalign*} %2.64
	& \boldsymbol{\pmast2\pmcdot64}. \quad \pmthm \pmdottt p \pmor q \pmdot \pmimp \pmdott p \pmor \pmnot q \pmdot \pmimp \pmdot p \quad \pmbr{\pmSUbb{\pmast2\pmcdot63}{q}{p}{p}{q}\pmdot\text{Perm}} & 
\end{flalign*}
\begin{flalign*} %2.65
	& \boldsymbol{\pmast2\pmcdot65}. \quad \pmthm \pmdottt p \pmimp q \pmdot \pmimp \pmdott p \pmimp \pmnot q \pmdot \pmimp \pmdot \pmnot p \quad \pmSub{\pmast2\pmcdot64}{\pmnot p}{p} & 
\end{flalign*}
\begin{flalign*} %2.67
	& \boldsymbol{\pmast2\pmcdot67}. \quad \pmthm \pmdottt p \pmor q \pmdot \pmimp \pmdot q \pmdott \pmimp \pmdot p \pmimp q & 
\end{flalign*}
\pmdemi
\begin{flalign*}%2.67
	&& & [\pmast2\pmcdot54\pmdot\text{Syll}] & &\pmthm \pmdottt p \pmor q \pmdot \pmimp \pmdot q \pmdott \pmimp \pmdott \pmnot p \pmimp q \pmdot \pmimp \pmdot q  & (1) \\
	&& & [\pmast2\pmcdot24\pmdot\text{Syll}] & &\pmthm \pmdottt \pmnot p \pmimp q \pmdot \pmimp \pmdot \pmdot q \pmdott \pmimp \pmdot p \pmimp q  & (2) \\
	&& & \multispan4{$\pmthm \pmdot (1)\pmdot(2)\pmdot\text{Syll}\pmdot \pmithm \pmdot \pmprop$\hfill} & &&
\end{flalign*}
\pageFsp{113}{108} 
\begin{flalign*} %2.68
	& \boldsymbol{\pmast2\pmcdot68}. \quad \pmthm \pmdottt p \pmimp q \pmdot \pmimp \pmdot q \pmdott \pmimp \pmdot p \pmor q & 
\end{flalign*}
\pmdemi
\begin{flalign*}%2.68
	&& & \pmSub{\pmast2\pmcdot67}{\pmnot p}{p} \quad \pmthm \pmdottt p \pmimp q \pmdot \pmimp \pmdot q \pmdott \pmimp \pmdot \pmnot p \pmimp q  & && (1) \\
	&& & \pmthm \pmdot (1)\pmdot\pmast2\pmcdot54\pmdot \pmithm \pmdot \pmprop & &&
\end{flalign*}
\begin{flushleft} %2.69
	\(\boldsymbol{\pmast2\pmcdot69}. \quad \pmthm \pmdottt p \pmimp q \pmdot \pmimp \pmdot q \pmdott \pmimp \pmdott q \pmimp p \pmdot \pmimp \pmdot p \quad \pmbr{\pmast2\pmcdot68\pmand\text{Perm}\pmand\pmSUbb{\pmast2\pmcdot62}{q}{p}{p}{q}}\)
\end{flushleft}
\begin{flalign*} %2.73
	& \boldsymbol{\pmast2\pmcdot73}. \quad \pmthm \pmdottt p \pmimp q \pmdot \pmimp \pmdott p \pmor q \pmor r \pmdot \pmimp \pmdot q \pmor r  \quad [\pmast2\pmcdot621\pmcdot38] & 
\end{flalign*}
\begin{flalign*} %2.74
	& \boldsymbol{\pmast2\pmcdot74}. \quad \pmthm \pmdottt q \pmimp p \pmdot \pmimp \pmdott p \pmor q \pmor r \pmdot \pmimp \pmdot p \pmor r \quad \pmbr{\pmSUbb{\pmast2\pmcdot73}{q}{p}{p}{q}\pmand\text{Assoc}\pmand\text{Syll}} & 
\end{flalign*}
\begin{flalign*} %2.75
	& \boldsymbol{\pmast2\pmcdot75}. \quad \pmthm \pmdotttt p \pmor q  \pmdot \pmimp \pmdottt p \pmdot \pmor \pmdot q \pmimp r \pmdott \pmimp \pmdot p \pmor r \quad \pmbr{\pmSUb{\pmast2\pmcdot74}{\pmnot q}{q}\pmand\pmast2\pmcdot53\pmcdot31} & 
\end{flalign*}
\begin{flalign*} %2.76
	& \boldsymbol{\pmast2\pmcdot76}. \quad \pmthm \pmdottt p \pmdot \pmor \pmdot q \pmimp r \pmdott \pmimp \pmdott p \pmor q \pmdot \pmimp \pmdot p \pmor r \quad [\pmast2\pmcdot75\pmand\text{Comm}] & 
\end{flalign*}
\begin{flalign*} %2.77
	& \boldsymbol{\pmast2\pmcdot77}. \quad \pmthm \pmdottt p \pmdot \pmimp \pmdot q \pmimp r \pmdott \pmimp \pmdott p \pmimp q \pmdot \pmimp \pmdot p \pmimp r \quad \pmSub{\pmast2\pmcdot76}{\pmnot p}{p} & 
\end{flalign*}
\begin{flalign*} %2.8
	& \boldsymbol{\pmast2\pmcdot8}. \quad \pmthm \pmdottt q \pmor r \pmdot \pmimp \pmdott \pmnot r \pmor s \pmdot \pmimp \pmdot q \pmor s & 
\end{flalign*}
\pmdemi
\begin{flalign*}%2.8
	&& [\pmast2\pmcdot53]\pmand\text{Perm}\pmdot \pmithm \pmdottt q \pmor r \pmdot\; &\pmimp \pmdott \pmnot r \pmimp q \pmdott  & && \\
	&& [\pmast2\pmcdot38] \qquad\qquad\qquad\quad\quad\quad & \pmimp \pmdott \pmnot r \pmor s \pmdot \pmimp \pmdot q \pmor s \pmdottt \pmithm \pmdot \pmprop &&&
\end{flalign*}
\begin{flalign*} %2.81
	& \boldsymbol{\pmast2\pmcdot81}. \quad \pmthm \pmdotttt q \pmdot \pmimp \pmdot r \pmimp s \pmdott \pmimp \pmdottt p \pmor q \pmdot \pmimp \pmdott p \pmor r \pmdot \pmimp \pmdot p \pmor s & 
\end{flalign*}
\pmdemi
\begin{flalign*}%2.81
	&& & \pmthm\pmdot\text{Sum}\pmdot \pmithm \pmdotttt q \pmdot \pmimp \pmdot r \pmimp s \pmdott \pmimp \pmdottt p \pmor q \pmdot \pmimp \pmdott p \pmdot \pmor \pmdot r \pmimp s  & && (1) \\
	&& & \pmthm \pmdot \pmast2\pmcdot76\pmand\text{Syll} \pmdot \pmithm \pmdotttt p \pmor q \pmdot \pmimp \pmdott p \pmdot \pmor \pmdot r \pmimp s \pmdottt \pmimp \pmdottt p \pmor q \pmdot \pmimp \pmdott p \pmor r \pmdot \pmimp \pmdot p \pmor s& && (2) \\
	&& & \pmthm \pmdot (1)\pmand(2) \pmdot \pmithm \pmdot \pmprop & &&
\end{flalign*}
\begin{flalign*} %2.82
	& \boldsymbol{\pmast2\pmcdot82}. \quad \pmthm \pmdottt p \pmor q \pmor r \pmdot \pmimp \pmdott p \pmor \pmnot r \pmor s \pmdot \pmimp \pmdot p \pmor q \pmor s \quad \pmbr{\pmast2\pmcdot8\pmand\pmSUbbb{\pmast2\pmcdot81}{q\pmor r}{q}{\pmnot r \pmor s}{r}{q\pmor s}{s}} & 
\end{flalign*}
\begin{flalign*} %2.83
	& \boldsymbol{\pmast2\pmcdot83}. \quad \pmthm \pmdotttt p \pmdot \pmimp \pmdot q \pmimp r \pmdott \pmimp \pmdottt p \pmdot \pmimp \pmdot r \pmimp s \pmdott \pmimp \pmdott p \pmdot \pmimp \pmdot q \pmimp s \quad \pmSubb{\pmast2\pmcdot82}{\pmnot p}{p}{\pmnot q}{q} & 
\end{flalign*}
\begin{flushleft} %2.85
	\(\boldsymbol{\pmast2\pmcdot85}. \quad \pmthm \pmdottt p \pmor q \pmdot \pmimp \pmdot p \pmor r \pmdott \pmimp \pmdott p \pmdot \pmor \pmdot q \pmimp r\)
\end{flushleft}
\pmdemi
\begin{flalign*}
	&& & \multispan3{$[\text{Add}\pmand\text{Syll}] \:\; \pmthm \pmdottt p \pmor q \pmdot \pmimp \pmdot r \pmdott \pmimp \pmdot q \pmimp r$ \hfill} & && (1) \\
	&& &\multispan3{$\pmthm\pmdot\pmast2\pmcdot55\pmdot\pmithm\pmdotttt \pmnot p \pmdot \pmimp \pmdottt p \pmor r \pmdot \pmimp \pmdot r \pmdottt$ \hfill} & &&\\
	&& & [\text{Syll}] && \pmimp \pmdottt p \pmor q \pmdot \pmimp \pmdot p \pmor r \pmdott \pmimp \pmdott p \pmor q \pmdot \pmimp \pmdot r \pmdottt & &&\\
	&& & [(1)\pmand\pmast2\pmcdot83] & & \pmimp \pmdottt p \pmor q \pmdot \pmimp \pmdot p \pmor r \pmdott \pmimp \pmdott q \pmimp r  & && (2) \\
	&& & \multispan3{$\pmthm\pmdot(2)\pmand\text{Comm} \pmdot \pmithm \pmdottt \hspace{1.1em} p \pmor q \pmdot \pmimp \pmdot p \pmor r \pmdott \pmimp \pmdott \pmnot p \pmdot \pmimp \pmdot q \pmimp r \pmdott $ \hfill} & && \\
	&& & [\pmast2\pmcdot54] & & \qquad \qquad \qquad \qquad \quad \; \pmimp \pmdott p \pmdot \pmor \pmdot q \pmimp r \pmdottt \pmithm \pmdot \pmprop & && 
\end{flalign*}
\begin{flalign*} %2.86
	& \boldsymbol{\pmast2\pmcdot86}. \quad \pmthm \pmdottt p \pmimp q \pmdot \pmimp \pmdot p \pmimp r \pmdott \pmimp \pmdott p \pmdot \pmimp \pmdot q \pmimp r \quad \pmSub{\pmast2\pmcdot85}{\pmnot p}{p} & 
\end{flalign*}
